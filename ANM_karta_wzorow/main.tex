\documentclass{article}
\usepackage[T1]{fontenc}
\usepackage[utf8]{inputenc}
\usepackage[polish]{babel}
\selectlanguage{polish}
\usepackage{lmodern}
\usepackage{amsmath}
\usepackage{amssymb}
\usepackage{amsfonts}
\usepackage{amsthm}
\usepackage{hyperref}
\usepackage{enumitem}
\hypersetup{
    colorlinks=true,
    linkcolor=blue,
    filecolor=magenta,      
    urlcolor=cyan
}
\usepackage{graphicx}
\usepackage{subcaption}
\usepackage[export]{adjustbox}
\usepackage{wrapfig}
\usepackage[nofoot,hdivide={0cm,*,1cm},vdivide={0cm,*,0cm}]{geometry}
\usepackage{multirow}
\usepackage[table]{xcolor}
\usepackage{sectsty}
\sectionfont{\fontsize{12}{8}\selectfont}
\subsectionfont{\fontsize{10}{8}\selectfont}
\usepackage[extreme]{savetrees}
\pagestyle{empty}
\usepackage{titlesec}
\titlespacing*{\section}{0pt}{0.1ex}{0.1ex}
\titlespacing*{\subsection}{0pt}{0.1ex}{0.1ex}
\titlespacing*{\subsubsection}{0pt}{0.1ex}{0.1ex}

\title{Ściomga}
\author{minion.chomiczewski }
\date{November 2022}
\begin{document}
\setlength{\abovedisplayskip}{0.0pt}
\setlength{\belowdisplayskip}{0.0pt}
\setlength{\tabcolsep}{0.0pt}
\begin{minipage}[t]{.325\textwidth}
\section*{Operacje arytmetyczne}
\subsection*{Twierdzenie o błędzie obliczeń}
Jeśli $|\alpha_{j}| \leq u$ i $\rho_j = \pm 1$ dla \\ $j = 1,2,...,n$ oraz
$nu < 1$, to zachodzi:\\
$\prod_{j=1}^{n}(1+\alpha_j)^{\rho_j} = 1 + \theta_n$
gdzie: $|\theta_n| \leq \gamma_n = \frac{nu}{1-nu} \approx nu.$

\subsection*{Twierdzenie o błędzie obliczeń 2}
Jeśli $|\alpha_j| \leq u$ dla $j = 1,2,...,n$ oraz $nu < 10^{-k}$, gdzie $k \in \mathbf{N}_{+}$ to zachodzi: $\prod_{j=1}^{n}(1+\alpha_j) = 1 + \eta_n$
gdzie $|\eta_n| \leq (1 + 10^{-k})nu$
Niech $X_{fl} := \{rd(x) : x \in X\}$. $u = \frac{1}{2}2^{-t}$ i jest to precyzja arytmetyki.
\section*{Uwarunkowanie zadania}
Jeśli niewielkie względne zmiany danych zadania powodują duże względne zmiany jego rozwiązania, to zadanie jest \color{red} \textbf{źle uwarunkowanym. } \color{black} Wielkości charakteryzujące wpływ zaburzeń danych na odkształcenia rozwiązania nazywamy \color{red} \textbf{wskaźnikiem uwarunkowania } \color{black} zadania.
\begin{equation*}
    \left| \frac{f(x+h) - f(x)}{f(x)} \right| \approx \left| \frac{hf'(x)}{f(x)}\right|
\end{equation*}
\begin{equation*}
    = \left| \frac{xf'(x)}{f(x)}\right|\left| \frac{h}{x} \right| = C_f(x)\cdot\left|\frac{h}{x}\right|
\end{equation*}
$C_f(x)$ jest \color{red}\textit{wskaźnik uwarunkowania}\color{black}.
\section*{Poprawność numeryczna}
%Algorytm \color{red}\textit{numerycznie poprawnym }\color{black} nazywamy taki algorytm, dla którego \textbf{obliczone rozwiązanie jest mało zaburzonym rozwiązaniem dokładnym dla mało zaburzonych danych. } Przez "małe zaburzenia" rozumiemy tu zaburzenia na poziomie błędu reprezentacji.\\
Niech $\overline{y} = Alg(x)$ tzn. $\overline{y}$ jest wynikiem działania pewnego algorytmu wtedy, jeśli dla pewnego $\Delta x,\Delta y$ zachodzi $\overline{y} + \Delta y = f(x + \Delta x)$ tzn. \textbf{wynik jest lekko zaburzony dla lekko zaburzonych danych } to algorytm jest \color{red}\textbf{numerycznie poprawny}\color{black}.\\
Jeśli $\Delta y = 0$ to algorytm jest \color{red} \textbf{numerycznie "bardzo"\: poprawny\:}\color{black} i to oznacza że
algorytm daje \textbf{wynik dla lekko zaburzonych danych.}
\section*{Równania nieliniowe}
\subsection*{Zera wielokrotne}
\textbf{Twierdzenie:\;} Jeśli $f \in C^m[a,b]$ oraz $a < \alpha < b$ i zachodzi:
\begin{equation*}
    f(\alpha) = f'(\alpha) = ... = f^{(m-1)}(\alpha) = 0
\end{equation*}
oraz $f^{(m)}(\alpha) \neq 0$ to $\alpha$ jest \color{red}\textbf{$m$-krotnym zerem funkcji $f$.}\color{black}
\subsection*{Analiza zbieżności}
Niech ciąg $\{a_k\}$ będzie zbieżny do $g$. Jeśli istnieją takie liczby rzeczywiste $p$ oraz $C,\;(C > 0)$, że 
\begin{equation*}
    \lim_{n \leftarrow \infty} \left|\frac{a_{n+1} - g}{(a_n - g)^p}\right| = C
\end{equation*}
to $p$ nazywamy \color{red}\textbf{wykładnikiem zbieżności ciągu},\color{black}\;a $C$ - \color{red}\textbf{stałą asymptotyczną błędu}\color{black}.
%Jeśli $p = 1$ oraz $C \in (0,1)$ to zbieżność jest \textbf{liniowa,\;} dla $p = 2$ jest \textbf{kwadratowa,\;} itd. \\
%Gdy $p = 1$ i $C = 1$ mówimy \\o zbieżności \textbf{podliniowej.\;} Natomiast jeśli $p = 1$ i $ C = 0$ to zbieżność tą nazywamy \textbf{nadliniową.}
\subsection*{Metoda Newtona}
\subsubsection*{Algorytm}
Niech $f \in C[a,b]$, wtedy budujemy ciąg ${x_n}$, taki że: $x_{n+1} = x_{n} + h_n$, gdzie $h_n := -\frac{f(x_n)}{f'(x_n)}$ dla $n \in \mathbf{N}$.
\subsubsection*{Zbieżność}
Niech $e_n := x_n - \alpha$, $F(x) = x - \frac{f(x)}{f'(x)}$, oraz $e_{n+1} = \frac{1}{2}F''(\eta_n)e_n^2$ gdzie $\eta_n \in interv(x_n,\alpha)$.
\textbf{Twierdzenie:\;} Jeśli przybliżenie $x_0$ jest dostatecznie bliskie pojedynczego zera $\alpha$ równania $f(x) = 0$ to metoda Newtona jest zbieżna kwadratowo do $\alpha$.  \textbf{Twierdzenie:\;} Jeśli $sgn(f'(x)) = sgn(f''(x)) \neq 0$ dla $x \in \mathbf{R}$, oraz $\alpha$ jest pojedynczym pierwiastkiem równania $f(x) = 0$. 
Wówczas $\alpha$ jest jedynym pierwiastkiem tego równania i
Metoda Newtona zbiega dla dowolnego początkowego $x_{0}$.
\textbf{Twierdzenie:\;} Jeśli $f \in C^2[a,b]$, $f'(x)f''(x) \neq 0$ dla dowolnego $x \in [a,b]$ i że $f(a)f(b) < 0$ oraz  $\left|\frac{f(a)}{f'(a)}\right| < b-a$, $\left|\frac{f(b)}{f'(b)}\right| < b-a$
\end{minipage}
\begin{minipage}[t]{.01\textwidth}
    |\\|\\|\\|\\|\\|\\|\\|\\|\\|\\|\\|\\|\\|\\|\\|\\|\\|\\|\\|\\|\\|\\|\\|\\|\\|\\|\\|\\|\\|\\|\\|\\|\\|\\|\\|\\|\\|\\|\\|\\|\\|\\|\\|\\|\\|\\|\\|\\|\\|\\|\\|\\|\\|\\|\\|\\|\\|\\|\\|\\|\\|\\|\\|\\|\\|\\|\\|\\|
\end{minipage}
\begin{minipage}[t]{.33\textwidth}
%\\
  to metoda Newotna jest zbieżna dla dowolnego $x_0 \in [a,b]$.
 \subsubsection*{Dla $r$-krotnego pierwiastka}
 $x_{n+1} = x_n + r_nh_n$, gdzie $h_n := -\frac{f(x_n)}{f'(x_n)}$, $r_n := \frac{x_{n-1}-x_{n-2}}{2x_{n-1}-x_n-x_{n-2}}$
\subsection*{Metoda siecznych}
Jest ona podobna do  metody Newtony, tylko że w miejscu $f'(x_n)$ używamy \textbf{ilorazu różnicowego.}
\begin{equation*}
    f[x_{n-1},x_n] := \frac{f(x_n)-f(x_{n-1})}{x_n - x_{n-1}}
\end{equation*}
\section*{Pierwiastki wielomianów}
\textbf{Twierdzenie:\;} - Wszystkie pierwiastki wielomianu leżą w kole otwartym o środku w punkcie 0 płaszczyzny zespolonej i promieniu $R := 1 + \frac{\max_{1 \leqslant k \leqslant n}|a_k|}{|a_0|}$
\subsection*{Metoda Leguerre'a}
$z_{k+1} = z_k - \frac{nw(z_k)}{w'(z_k) \pm \sqrt{H(z_k)}}$
$H(x) := (n-1)[(n-1)w'^2(x) - nw(x)w''(x)]$\\
Znak $\pm$ należy wybrać by $|z_{k+1} - z_{k}|$ było jak najmniejsze. Algorytm ten jest zbieżny \textbf{sześciennie\;} do pierwiastków pojedynczych rzeczywistych bądź zespolonych. Metoda ta jest zbieżna \textbf{globalnie\;} gdy wielomian ma tylko pierwiastki rzeczywiste.
%\subsection*{Metoda Bairstow'a}
%Rozpatrzmy wielomian:\\
%$p(z) = a_nz^n + a_{n-1}z^{n-1} + ... + a_0$\\
%Podzielmy go przez wielomian o stopniu 2 postaci: $z^2 - uz - v$, wtedy %otrzymamy następujące wielomiany:\\
%$q(z) = b_nz^{n-2} + b_{n-1}z^{n-3} + ... + b_3z + b_2$\\
%$r(z) = b_1(z-u) + b_0$\\
%Współczynniki można rekurencyjnie liczyć przy pomocy wzoru:\\
%$b_{n+1} = b_{n+2} = 0, b_k = a_k + ub_{k+1} + vb_{k+2}\;\; (n \geqslant %k \geqslant 0)$\\
%Niech $c_k = \frac{\partial b_k}{\partial u}$ i $d_k = \frac{\partial %b_{k-1}}{\partial v}$, wtedy zachodzą następujące zależności rekurencyjne:\\
%$c_k = b_{k+1} + uc_{k+1} + vc_{k+2}\;\;(c_{n} = c_{n+1} = 0)$\\
%$d_k = b_{k+1} + ud_{k+1} + vd_{k+2}\;\;(d_{n} = d_{n+1} = 0)$\\
%Z tego można wywnioskować, że $c_k = d_k$
%\subsubsection*{Idea metody}
%szukamy takich $u,v$ by zachodziło $b_1 = b_0 = 0$, zatem musimy rozwiązać następujące dwa równania:\\
%$b_1(u,v) = 0, b_0(u,v) = 0$\\
%Są to równania nieliniowe ze względu na $u,v$ zatem korzystamy z metody Newtona aby je znaleźć. W każdym kroku iteracji szukamy takich poprawek $h^{(u)}_n,h^{(v)}_n$, by zachodziły
%następujące dwa równania:\\
%$b_0(u_n,v_n) + c_0(u_n,v_n)h^{(u)}_n + c_1(u_n,v_n)h^{(v)}_n = 0$ \\
%$b_1(u_n,v_n) + c_1(u_n,v_n)h^{(u)}_n + c_2(u_n,v_n)h^{(v)}_n = 0$ \\
%stąd można otrzymać: $h_{n}^{(u)} = \frac{c_1b_1 - c_2b_0}{J},\; %h_{n}^{(v)} = \frac{c_1b_0 - c_0b_1}{J},\; J = c_0c_2-c_1^2$
\section*{Interpolacja}
Mając dane: $[x_0,x_1,...,x_n]$ oraz $[y_0,y_1,...,y_n]$, znaleźć taki wielomian
$L_n(x)$ stopnia conajwyżej $n$ o własnościach: $L_n(x_i) = y_i, \;\; i = 0,1,...,n$
\subsection*{Postać Lagrange'a}
$L_n(x) = \sum_{k=0}^ny_k\lambda_k(x)$, gdzie: \\$\lambda_k(x) = \prod_{j=0,j\neq k}^n\frac{x-x_j}{x_k-x_j}$\\
Wielomian ten jest, wielomianem najniższego stopnia interpolującym zadane punkty.
\subsubsection*{Postać barycentryczna}
Niech $\sigma_k = \prod_{j=0,j \neq k}^n\frac{1}{x_k - x_j}$, wtedy:
\begin{equation*}
    L_n(x) = \begin{cases}
        \frac{\sum_{k=0}^n\frac{\sigma_k}{x-x_k}y_k}{\sum_{k=0}^n\frac{\sigma_k}{x-x_k}}, & \text{gdy } x \notin X\\
        y_k, & \text{wpp.}
    \end{cases}
\end{equation*}
gdzie $X = \{x_0,x_1,...,x_n\}$.
\subsubsection*{wariant wzoru Lagrange'a}
Ponieważ $\sigma_k = \frac{1}{p'_{n+1}(x_k)}$ to $L_n(x) = p_{n+1}(x)\sum_{k=0}^ny_k\frac{\sigma_k}{x-x_k}$
\subsubsection*{Twierdzenia}
Niech wielomian $L_n(x)$ postaci Langrange interpoluje funkcje $f$ w $n+1$ węzłach wtedy następujące równości są prawdziwe:
$\sum_{k=0}^n\lambda_k(x) = 1$ \\
$\sum_{k=0}^n\lambda_k(0)x^j_k = \begin{cases}
        1 & j = 0, \\
        0 & j \in \mathbf{N_{\leq n}}.
    \end{cases}$
\subsection*{Postać Newtona}
Niech $p_0(x) = 1, p_k = \prod_{i=0}^{k-1}(x-x_i) \text{ gdy } k $
$= 1,2,...,n+1$ oraz $b_k = \sum_{i=0}^k\frac{y_i}{p'_{k+1}(x_i)} = \sum_{i=0}^k\frac{y_i}{\prod_{j=0,j\neq i}^k(x_i - x_j)}$ gdy $ k = 0,1,...,n$.
\\Wtedy \color{red}$L_n(x) = \sum_{k=0}^nb_kp_k(x)$\color{black}
\subsection*{Twierdzenie błędu interpolacji}
Jeśli $f \in C^{n+1}[a,b]$, a wielomian $L_n \in \Pi_n$ jest 
wielomianem interpolującym $f$ w $n+1$ węzłach w przedziale $[a,b]$, to dla każdego $x \in [a,b]$
zachodzi równość $f(x) - L_n(x) = \frac{f^{(n+1)(\xi_x)}}{(n+1)!}p_{n+1}(x)$
gdzie $\xi_x$ jest pewną liczbą zależną od $x$ z przedział $[a,b]$
\end{minipage}
\begin{minipage}[t]{.005\textwidth}
    |\\|\\|\\|\\|\\|\\|\\|\\|\\|\\|\\|\\|\\|\\|\\|\\|\\|\\|\\|\\|\\|\\|\\|\\|\\|\\|\\|\\|\\|\\|\\|\\|\\|\\|\\|\\|\\|\\|\\|\\|\\|\\|\\|\\|\\|\\|\\|\\|\\|\\|\\|\\|\\|\\|\\|\\|\\|\\|\\|\\|\\|\\|\\|\\|\\|\\|\\|\\|
\end{minipage}
\begin{minipage}[t]{.33\textwidth}
\subsection*{Maksymalny błąd interpolacji}
Jeśli $f \in C^{n+1}[-1,1]$ to $||f(x) - L_n(x) ||^{[-1,1]}_{\infty} \leq \frac{M_{n+1}P_{n+1}}{(n+1)!}$, gdzie $M_{n+1} = ||f^{(n+1)}(x)||^{[-1,1]}_{\infty}$, \\a $P_{n+1} = ||p_{n+1}(x)||^{[-1,1]}_{\infty}$.
\subsection*{Iloraz różnicowy}
\begin{equation*}
    p[x_0,x_1,\dotsc,x_n] \stackrel{def}{=} \sum_{i=0}^{n}\frac{f(x_i)}{\prod_{\substack{j=0 \\ j\neq i}}^{n} (x_i-x_j)} 
\end{equation*}
\begin{equation*}
    p[x_0,...,x_n] \stackrel{def}{=} \frac{f[x_1,...,x_n] - f[x_0,...,x_{n-1}]}{x_n - x_0}
\end{equation*}
\begin{equation*}
    p[\stackrel{\text{r razy}}{x_0,...,x_0}] \stackrel{def}{=} \frac{1}{(r-1)!} f^{(r-1)}(x_0)
\end{equation*}
\subsection*{Wielomiany Czybyszewa}
\subsubsection*{Definicja}
\begin{gather*}
    T_0(x) = 1,\text{ }T_1(x) = x, \\ T_n(x) = 2xT_{n-1}(x) - T_{n-2}(x)\\
    T_n(x) = \cos{(n \arccos{(x)})}
\end{gather*}
\subsubsection*{Ekstrema i miejsca zerowe}
Dla zadanego $T_k(x)$ punkty ekstremalne to $u_{kj} = \cos{(\frac{j\pi}{k})}$
dla $j = 0...k$ natomiast miejsca zerowe to $t_{kj} = \cos{\frac{(2j+1)\pi}{2k}}$ dla $j = 0...(k-1)$
\subsubsection*{Twierdzenie}
Każdy wielomian $w \in \Pi_n$ można jednoznacznie przedstawić w postaci 
$w(x) = \sum_{k=0}^{n}\textbf{'}c_kT_k(x)$ 
\subsubsection*{Twierdzenie 2}
Wielomian $\Tilde{T_n}(x) := 2^{(1-n)}T_n(x)$  ma najmniejszą normę w przedziale $[-1,1]$ spośród wszystkich wielomianów stopnia $\leq n$, o współczynniku wiądącym $1$.
\subsubsection*{Interpolacja}
Wielomian $I_n \in \Pi_n$ interpolujący funkcje $f$ w węzłach $t_j = t_{n+1,j} = \cos{\frac{2j+1}{2n+2}\pi}$
można zapisać w postaci $\sum_{k=0}^{n} \textbf{'} \alpha_k T_k(x)$, gdzie $a_k = \frac{2}{n+1}\sum_{j=0}^{n}f(t_j)T_k(t_j)$.
Wielomian $J_n \in \Pi_n$ interpolujący funkcje $f$ w węzłach $u_j = u_{n,j} = \cos{\frac{\pi j}{n}}$ można zapisać wzorem $\sum_{k=0}^{n}\textbf{''}\beta_kT_k(x)$
gdzie $\beta_k = \frac{2}{n}\sum\limits_{j=0}^{n}\textbf{''}f(u_j)T_k(u_j)$
\subsection*{Twierdzenia}
\subsubsection*{Fabera}
Dla każdej tablicy węzłów $\{x_{nk}\}$ istnieje taka funkcja ciągła w przedzialie $[a,b]$, do której ciąg wielomianów interpolacyjnych nie jest zbieżny jednostajnie (tj. taka, że $\max\limits_{a\leq x \leq b} |f(x) - L_n(x)| \nrightarrow 0).$
\subsubsection*{Kryłow}
Niech dana będzie funkcja $f \in C^1[-1,1]$ i niech $\{L_n\}$ będzie ciągiem wielomianów interpolujących funkcję $f$ w węzłach Czybyszewa. Wówczas dla $\forall x \in [-1,1]$ jest $\lim\limits_{n\rightarrow\infty}L_n(x) = f(x)$.
\subsection*{Funkcje sklejane}
\subsubsection*{Typy funkcji}
naturalna - $s''(a) = s''(b) = 0$ \\
zupełna  - $s'(a) = f'(a) \text{ i } s'(b) = f'(b)$ \\
okresowa - $s'(a) = s'(b) \text{ oraz } s''(a) = s''(b)$ jeśli $f$ jest okresowa z okresem $b-a$

\subsubsection*{Wzór dla naturalnej}
Definiujemy następujące wartości $h_k = x_k - x_{k-1}$, $\lambda_k = \frac{h_k}{h_k + h_{k+1}}$,  $M_0 = M_n = 0, \lambda_k M_{k-1} + 2M_k + (1-\lambda_k) M_k+1=6f[x_{k-1},x_k,x_{k+1}]$ wtedy dla każdego przedziału $[x_{k-1},x_k]$ $s(x) = h_k^{-1}[\frac{1}{6}M_{k-1}(x_k - x)^3 + \frac{1}{6}M_k(x-x_{k-1})^3 + (f(x_{k-1}) - \frac{1}{6}M_{k-1}h_k^2)(x_k -x) + (f(x_k) - \frac{1}{6}M_kh_k^2)(x-x_{k-1})]$
\end{minipage}
\newpage
\setlength{\abovedisplayskip}{0.0pt}
\setlength{\belowdisplayskip}{0.0pt}
\setlength{\tabcolsep}{0.0pt}
\begin{minipage}[t]{.33\textwidth}

\subsubsection*{Twierdzenie Holladay-a}
W klasie funkcji $F\in C^2[a,b]$ oraz spełniających $F(x_k) = y_k$ najmniejsza wartość całki $\int_a^b[F''(x)]^2 dx$ ma funkcja naturalna funkcja sklejana $III$-stopnia $s$. Przy tym $\int_a^b[s''(x)]^2dx = \sum_{k=1}^{n-1}(f[x_k,x_{k+1}]-f[x_{k-1},x_{k}])M_k$.
\subsubsection*{Obliczanie $M_k$}
Algorytm $q_0 = u_0 = 0, p_k = \lambda_k q_{k-1} + 2, q_k = \frac{(\lambda_k - 1)}{p_k},u_k = \frac{d_k -\lambda_k u_{k-1}}{p_k},d_k = 6f[x_{k-1},x_k,x_{k+1}],M_{n-1}=u_{n-1},$ $M_k = u_k + q_k M_{k+1}$
\section*{Aproksymacja}
\subsection*{Norma}
Wzór $\langle f,g \rangle := \int_a^b p(x)f(x)g(x)dx$ definiuje \textbf{iloczyn skalarny} funkcji $f,g \in C_p[a,b]$.\\
\textbf{Norma średniokwadratowa} $||f||_2 = \sqrt{\int_a^b p(x)f^2(x)dx}$. \textbf{Norma jednostajna} $||f||_\infty^T = \sup_T |f(x)|$
\subsection*{Ortogonalizacja}
\subsubsection*{Gram-Schmidt}
\begin{equation*}
    \begin{cases}
        g_1 := f_1 \\
        g_k := f_k - \sum_{i=1}^{k-1}\frac{\langle f_k,g_i\rangle}{\langle g_i, g_i\rangle}g_i & (k=2,3,...,m)
    \end{cases}
\end{equation*}
\subsubsection*{Wielomiany Ortogonalne}
$\overline{P_0}(x) = 1\text{, } \overline{P_1}(x) = x - c_1\text{, } \overline{P_k}(x) = (x-c_k) \overline{P_{k-1}}(x) -d_k\overline{P_{k-2}}(x)\text{, }c_k = \frac{\langle x\overline{P_{k-1}},\overline{P_{k-1}}\rangle}{\langle\overline{P_{k-1}},\overline{P_{k-1}}\rangle}$ $d_k = \frac{\langle\overline{P_{k-1}},\overline{P_{k-1}}\rangle}{\langle\overline{P_{k-2}},\overline{P_{k-2}}\rangle}$,jeśli $p(x)$ jest funkcją parzystą na przedziale $[-a,a]$ to $\forall_k c_k = 0$, oraz $\overline{P_{2m}}(x)$ jest funkcją parzystą, a $\overline{P_{2m+1}}(x)$ jest nieparzystą dla każdego $m \in \mathbf{N}$ 
\subsection*{Wielomian optymalny}
$w_n^* = \sum_{k=0}^{n}\frac{\langle f,P_k \rangle}{\langle P_k,P_k \rangle}P_k$,  natomiast $n$-ty błąd aproksymacji jest równy 
$||f-w_n^*||_2 = \sqrt{||f||_2^2 - \sum_{k=0}^{n}\frac{\langle f,P_k \rangle^2}{\langle P_k,P_k \rangle}}$
\subsection*{Algorytm Clenshawa}
Niech $s_n = \sum_{k=0}^{n}a_kP_k$ oraz niech $P_0 = \alpha_0\text{, } P_1 = (\alpha_1 x - \beta_1)P_0\text{, } P_k = (\alpha_k x - \beta_k) P_{k-1} - \gamma_k P_{k-2}$, wtedy $a_k = V_k - (\alpha_{k+1}x - \beta_{k+1})V_{k+1} + \gamma_{k+2}V_{k+2}$, gdzie $V_{n+1} = V_{n+2} = 0$. Wynikiem tego algoytmu jest $s_n(x) = \alpha_0 V_0$.
\subsection*{Twierdzenie o alternansie}
Niech $T$ będzie dowolnym podzbiorem domkniętym przedziału $[a,b]$. Na to, by wielomian $w_n$ był $n$-tym wielomianem optymalnym dla funkcji $f \in \mathbf{C}(T)$ potrzeba i wystarczy, żeby istniały takie punkty $x_0,x_2,...,x_{n+1} \in T$, że dla $e_n := f - w_n$ jest $e_n(x_k) = -e_n(x_{k-1})$ oraz $|e_n(x_j)| = ||e_n|_\infty^T$. Zbiór punktów $x_0,x_1,...,x_{n+1}$, w których $||e_n||_\infty^T = \max_{x \in T}|e_n(x)|$ z naprzemiennymi znakami nazywamy $n$-tym alternansem funkcji $f$.
\section*{Kwadratura}
\subsection*{Metoda punkta środkowego}
$\int_a^bf(x)dx \approx (b-a)f(\frac{a+b}{2})$
\subsection*{Metoda trapezów}
$\int_a^bf(x)dx \approx (b-a)\frac{f(a) + f(b)}{2}$, złożony wzór $T_n(f) := h\sum_{k=0}^n \text{''}f(t_k)$, $R_n^T(f) = -(b-a)\frac{h^2}{12}f\text{''}(\xi)$, gdzie $\xi \in (a,b)$
\subsection*{Metoda Simpsona}
$\int_a^bf(x)dx \approx \frac{1}{6}(f(a) + 4f(\frac{a+b}{2}) + f(b))$, złożony wzór Simpsona 
\end{minipage} 
\begin{minipage}[t]{.005\textwidth}
|\\|\\|\\|\\|\\|\\|\\|\\|\\|\\|\\|\\|\\|\\|\\|\\|\\|\\|\\|\\|\\|\\|\\|\\|\\|\\|\\|\\|\\|\\|\\|\\|\\|\\|\\|\\|\\|\\|\\|\\|\\|\\|\\|\\|\\|\\|\\|\\|\\|\\|\\|\\|\\|\\|\\|\\|\\|\\|\\|\\|\\|\\|\\|\\|\\|\\|\\|\\|
\end{minipage}
\begin{minipage}[t]{.33\textwidth}


 $S_n(f) := \frac{h}{3}(2\sum_{k=0}^m\text{''}f(t_{2k}) + 4\sum_{k=1}^mf(t_{2k-1})) = \frac{1}{3}(4T_n - T_m)$ gdzie $n = 2m$, $R_n^S(f) = -(b-a)\frac{h^4}{180}f^{(4)}(\eta)$, gdzie $\eta \in (a,b)$.
\subsection*{Metoda Romberga}
$T_{0,j} = T_j $ (Metoda trapezów), $T_{i,j+1} = \frac{4^{j+1}T_{i,j}-T_{i-1,j}}{4^{j+1} - 1}$
\subsection*{Rząd metody}
Niech $Q(f) = \sum_{k=0}^nA_kf(x_k)$, wtedy $R_n(f) = I_p(f) - Q(f)$ wtedy $Q_n$ jest rzędu $r$, jeśli $\forall_{f\in \Pi_{r-1}} R_n(f) = 0$ oraz $\exists_{w \in \Pi_{r}/ \Pi_{r-1} } R_n(w) \neq 0$. \textbf{Twierdzenie} Rząd $Q_n \leq 2n+2$.
\subsection*{Kwadratury interpolacyjne}
$Q_n(f) = \sum_{k=0}^n A_k f(x_k) \approx \int_a^b p(x)f(x)dx$, $A_k = \int_a^b p(x) \lambda_k(x)dx$, \textbf{Twierdzenie} Jeśli rząd $Q_n \leq n+1$ to $Q_n$ jest kw. interpolacyjną. \textbf{Twierdzenie Jacobego} Rząd $Q_n \leq n+1+m$ gdzie $1 \leq m \leq n+1$ wtw gdy spełnione są następujące warunki: $Q_n$ jest interpolacyjny oraz $\forall_{u \in \Pi_{m-1}} I_p(w*u) = 0 \text{  gdzie } w(x) = (x-x_0)(x-x_1)...(x-x_n)$ \textbf{Twierdzenie} Aby rząd był $2n+2$, $x_n$ muszą być zeremi $(n+1)$-ego wielomianu optymalnego.
\subsubsection*{Kwadratura Gaussa}
$Q_n(f) := \sum_{k=0}^nA_k^{(n)}f(x_k^{(n)})$, $A_k^{(n)} = \int_a^b p(x)\lambda_k(x)dx$, $\lambda_k(x) = \frac{w(x)}{w'(x_k)(x-x_k)}$, $w(x) = \overline{P}_{n+1}(x)$. \textbf{Lemat} $A_k^{(n)} = \frac{a_{n+1}}{a_n}\frac{||P_n||^2}{P'_{n+1}(x_k)P_n(x_k)}$. \textbf{Lemat} $A_k^{(n)} > 0$. \textbf{Lemat} $R_n(f) = \frac{f^{(2n+2)}(\xi)}{(2n+2)!a_{n+1}^2}\int_a^bp(x)[P_{n+1}(x)]^2dx$, \textbf{Lemat} Jeśli $f\in \mathbf{C}[a,b]$ to $\lim_{n\rightarrow\infty}Q_n = I_p(f)$
\subsubsection*{Gauss-Czybyszew}
$I_p(f) = \int_{-1}^{1}\frac{1}{\sqrt{1-x^2}}f(x)dx$, $Q^{GC}_n := \frac{1}{\sqrt{1-x^2}}I_n(x)dx$, $I_n(t_k) = f(t_k)$, $t_k = \cos{\frac{2k+1}{2n+2}\pi}$, $I_n(x) = \sum_{i=0}^n\text{`}\alpha_i T_i(x)$, $\alpha_i := \frac{2}{n+1}\sum_{k=0}^nf(t_k)T_i(t_k)$, $Q_n^{GC} := \sum_{k=0}^n A_kf(t_k)$, $A_k = \frac{\pi}{n+1}$
\subsubsection*{Kwadratura Lobatto}
$I_p(f) = \int_{-1}^{1}\frac{1}{\sqrt{1-x^2}}f(x)dx$, $Q_n^L = \int_{-1}^{1}\frac{1}{\sqrt{1-x^2}}J_n(x)dx$, $J_n(u_k) = f(u_k), u_k = \cos{\frac{k}{n}\pi}$, $J_n(x) = \sum_{j=0}^n\text{''}\beta_jT_j(x)$, $\beta_j = \frac{2}{n}\sum_{k=0}^n\text{''}f(u_k)T_j(u_k)$, $Q_n^L(f) = \sum_{k=0}^n\text{''}A_kf(u_k), A_k = \frac{\pi}{n}$, \textbf{Rząd $2n$}.
\subsubsection*{Clenshaw-Curtis}
$I(f) = \int_{-1}^1 f(x) dx$, $Q_n^{CC}(f) = \sum_{k=0}^n\text{''}A_k^{(n)}f(u_k),A_k^{(n)}:=\frac{4}{n}\sum_{j=0}^{\frac{n}{2}}\text{''}\frac{T_{2j}(u_k)}{1-4j^2}$, jeśli $n$ jest nieparzyste to powinien być jeden ` w wzorze na $A_k$
\section*{Algebra}
\subsection*{Normy wektorowe}
$||x||_1 = \sum_{i=1}^n |x_i|$, $||x||_2 = \sqrt{\sum_{i=1}^n x_i^2}$, $||x||_\infty = \max_{1\leq i \leq n} |x_i|$
\subsection*{Norma macierzowa}
\textbf{Definicja}Musi ona spełnić te same własności co norma oraz \textit{podmultiplikatywnośś} tzn. $||AB|| \leq ||A||\cdot||B||$
\textbf{Normy indukowane} $||A|| := \sup_{x\neq 0}\frac{||Ax||}{||x||} := \max_{||x|| = 1}||Ax||$. 
Przykłady $||A||_1 = \max_{1\leq j \leq n}\sum_{i=1}^n|a_{i,j}|$, $||A||_\infty = \max_{1\leq i \leq n}\sum_{j=1}^n|a_{i,j}|$, $||A||_2 = \sqrt{\rho(A^TA)}$, $\rho$ - największa wartość własna.  $||A||_2$ nazywana jest również \textit{normą spektralną}.\textbf{Normy zgodne} $\forall_{A\in R^{n\times n}}\forall_{x \in R^n} ||Ax|| \leq ||A||\cdot||x||$
\end{minipage}
\begin{minipage}[t]{.005\textwidth}
|\\|\\|\\|\\|\\|\\|\\|\\|\\|\\|\\|\\|\\|\\|\\|\\|\\|\\|\\|\\|\\|\\|\\|\\|\\|\\|\\|\\|\\|\\|\\|\\|\\|\\|\\|\\|\\|\\|\\|\\|\\|\\|\\|\\|\\|\\|\\|\\|\\|\\|\\|\\|\\|\\|\\|\\|\\|\\|\\|\\|\\|\\|\\|\\|\\|\\|\\|\\|
\end{minipage}
\begin{minipage}[t]{.33\textwidth}
\subsection*{Rozkład LU}
Niech $A = [a_{i,j}]\in R^{n\times n}$ oraz nie jest ona osobliwa to istnieje dokładnie jedna para macierzy $L \in \mathbb{L}_n^{(1)}$ oraz $U \in \mathbb{U}_n^{(1)}$ taka że $LU = A$, oraz $det(A) = u_{11}u_{22}...u_{nn}$. \textbf{Rozkład:} dla $i=1,2,...,n$ $u_{i,j} = a_{i,j} - \sum_{k=1}^{i-1}l_{i,k}u_{k,j}$, $j = i,i+1,...,n$, $l_{j,i} = \frac{1}{u_{i,i}}\left( a_{j,i} - \sum_{k=1}^{i-1}l_{j,k}u_{k,i} \right)$ \textbf{Rozwiązanie} Jeśli $Ax =b$ to można rozwiązać układ $Ly = b$, $Ux = y$
\subsection*{Eliminacja Gaussa}
Rozważmy $Ax=b$, niech $A^{(1)}=[a_{i,j}^{(1)}] := A$, $b^{(1)} = [b_1^{(1)},...,b_n^{(1)}]^T := b$ wtedy $\sum_{j=k}^n a_{k,j}^{(k)}x_j = b_k^{(k)}$, gdzie $a_k^{(k)}\neq 0$ oraz $a_{i,j}^{(k)} = a_{i,j}^{(k-1)} + m_{i,k-1}a_{k-1,j}^{(k-1)}$, $b_i^{(k)} = b_{i}^{(k-1)} + m_{i,k-1}b_{k-1}^{(k-1)}$, $m_{i,k-1} = -\frac{a^{(k-1)}_{i,k-1}}{a^{(k-1)}_{k-1,k-1}}$ 
\textbf{Element główny} $a_{k,k}^{(k)}$, dla $k=1,2,...,n$ \textbf{Współczynnik wzrostu} $g_n := \frac{\max_{1\leq i,j,r \leq n} |a_{i,j}^{(r)}|}{\max_{1\leq i,j \leq n}|a_{i,j}|}$. Dla \underline{eliminacji z częściowym wyborem} elementu głównego zachodzi $g_m \leq 2^{n-1}$. Dla \underline{pełnego} wyboru ele. głównego zachodzi $g_n \leq \phi(n) = \sqrt{n}\sqrt{2^13^{\frac{1}{2}}...n^{\frac{1}{n-1}}} < 1.8n^{\frac{1}{2}+\log{\frac{n}{4}}}$ \textbf{Twierdzenie}: Niech $\overline{x}$ oznacz rozwiązanie układu $Ax =b$ w $t$-cyfrowej arytmetyce $fl$ za pomocą metody eliminacji z wyborem. Wówczas istnieje macierz $\delta A\in R^{n\times n}$ spełniająca $||\delta A||_\infty \leq Cn^3g_n2^{-t}||A||_\infty$ i taka, że $(A+\delta A)\overline{x} = b$.
\subsection*{Twierdzenie}
Rozważmy $Ax =b$ oraz $(A+\delta A)(x+\delta x) = b + \delta b$, gdzie $\delta A$i $\delta b$ są zaburzeniami macierzy $A$ i wektora $b$. Załóżmy, że $\eta = ||\delta A|| \cdot ||A^{-1}|| = \text{cond}(A)\frac{||\delta A||}{||A||} < 1$. Wówczas dla dowolnej pary norm zgodnych zachodzi $\frac{||\delta x||}{||x||}\leq \frac{\text{cond}(A)}{1-\eta}\left(\frac{||\delta b||}{||b||} + \frac{||\delta A||}{||A||}\right)$, gdzie $\text{cond}(A) = ||A||\cdot ||A^{-1}||$
\subsection*{Metoda Richardsona}
Definicja: $x^{(k+1)}=B_{\tau}x^{(k)} + c$ gdzie $B_\tau := I - \tau A$, $c:= \tau b$, $x_i^{(k+1)}=x_i^{(k)} + \tau\left(b_i - \sum_{j=1}^na_{i,j}x_j^{(k)}\right)$
\subsection*{Metoda Jacobiego}
$B = B_J := -D^{-1}(L+U)$, wersję skalarną opisuje wzór $x_i^{(k+1)}=\frac{1}{a_{i,i}}(b_i - \sum_{j=1,j\neq i}^na_{i,j}x_j^{(k)} = x_i^{(k)} + \frac{1}{a_{i,i}}\left(b_i - \sum_{j=1}^na_{i,j}x_j^{(k)}\right)$. Jeśli $A$ jest macierzą ze ściśle dominującą przekątną to $||B_J||_\infty < 1$ i metoda Jacobiego jest zbieżna.
\subsection*{Metoda Gaussa-Seidela}
$B_S := -(D+L)^{-1}U$, $x_i^{(k+1)}=x_i^{(k)}+\frac{1}{a_{i,i}}\left(b_i - \sum_{j=1}^{i-1}a_{i,j}x_j^{(k+1)} - \sum_{i=1}^na_{i,j}x_j^{(k)}\right)$
\subsection*{Metoda relaksacyjna}
$B_\omega := (I-\omega M)^{-1}(\omega N + (1-\omega)I)$, gdzie $M := -D^{-1}L$, $N:=-D^{-1}U$. $x_i^{(k+1)}=x_i^{(k)}+\frac{\omega}{a_{i,i}}\left(b_i - \sum_{j=1}^{i-1}a_{i,j}x_j^{(k+1)} - \sum_{i=1}^na_{i,j}x_j^{(k)}\right)$. Dla dowolnej nieosobliwej macierz $A$ zachodzi $\rho(B_\omega) \leq |\omega -1|$. Jeśli  macierz $A$ jest symetryczna i dodatnio określona to metoda relaksacyjna jest zbieżna dla każdego $\omega \in (0,2)$. Jeśli $A$ jest macierzą symetryczną, dodatnio określoną i niech ma postać blokowo-trójprzekątniową to $\omega_{opt} = \frac{2}{1+\sqrt{1-\rho(B_S)}}$. Optymalną wartością $\rho(B_\omega) = \omega_{opt}-1$
\end{minipage}
\end{document}